%---------- Inleiding ---------------------------------------------------------

% TODO: Is dit voorstel gebaseerd op een paper van Research Methods die je
% vorig jaar hebt ingediend? Heb je daarbij eventueel samengewerkt met een
% andere student?
% Zo ja, haal dan de tekst hieronder uit commentaar en pas aan.

%\paragraph{Opmerking}

% Dit voorstel is gebaseerd op het onderzoeksvoorstel dat werd geschreven in het
% kader van het vak Research Methods dat ik (vorig/dit) academiejaar heb
% uitgewerkt (met medesturent VOORNAAM NAAM als mede-auteur).
% 

\section{Inleiding}%
\label{sec:inleiding}

%Waarover zal je bachelorproef gaan? Introduceer het thema en zorg dat volgende zaken zeker duidelijk aanwezig zijn:

%\begin{itemize}
%  \item kaderen thema
%  \item de doelgroep
%  \item de probleemstelling en (centrale) onderzoeksvraag
%  \item de onderzoeksdoelstelling
%\end{itemize}


In de huidige IT-sector in Europa ervaart ongeveer 61\% van de werknemers stress door de hoge werkdruk \autocite{ISACA2025}. Deze stress kan een burnout veroorzaken en dat grotere gevolgen dan alleen de gezondheid van de persoon. Aangezien 78\% van de werknemers van het rapport aangeeft dat de werkgerelateerde stressfactoren hen belemmeren in het bijscholen van hun vaardigheden \autocite{Staff2025} . Dit kan hun vermogen beperken om skills te ontwikkelen die zouden kunnen bijdragen aan een productievere teamomgeving.

\vspace{1em}Voor deze bachelorproef wordt de focus gelegd op 1 groot bedrijf: Cisco, waar ongeveer de helft van de medewerkers aangeeft meer dan 70 uur per week te werken \autocite{Xavier2025}. In september 2021 vroeg het bedrijf aan 100 werknemers, die een hogere functie beoefenen, hoe het gaat met hun mentale gezondheid. 10 van de 100 werknemers zegt dat hij of zij extra tijd moest nemen om te herstellen na een periode gewerkt te hebben. De top 2 oorzaken zijn in rechtstreeks verband met het probleem dat in deze proef behandeld wordt \autocite{Burton2021}. 

\vspace{1em}De doelgroep bestaat uit IT-professionals binnen deze organisatie die dagelijks geconfronteerd worden met hoge stress door een te groot takenpakket. Door het probleem te kaderen binnen 1 van de meest stressvolle bedrijven voor IT'ers, wordt het mogelijk een praktische en algemene oplossing te ontwikkelen.

\vspace{1em}De kern van het probleem is dat de huidige taakverdeling te weinig rekening houdt met de individuele capaciteiten, werkdruk en mentale belasting van werknemers. De centrale onderzoeksvraag luidt dan ook: ''Hoe kan artificiële intelligentie worden ingezet om de taakverdeling binnen Cisco te optimaliseren om zo enerzijds de mentale problemen bij IT'ers te verminderen als anderzijds de werking van het bedrijf vlotter te laten verlopen?''

\vspace{1em}De doelstelling van dit onderzoek is het uitwerken van een concept voor een AI-gestuurd systeem dat taken dynamisch kan toewijzen op basis van de competenties en de actuele mentale belasting van werknemers. Het eindresultaat is een goed onderbouwd ontwerp dat inzicht geeft in hoe een persoonlijk afgestemd takenpakket de werkdruk kan verminderen en daardoor de mentale gezondheid kan ondersteunen. 

%Denk er aan: een typische bachelorproef is \textit{toegepast onderzoek}, wat betekent dat je start vanuit een concrete probleemsituatie in bedrijfscontext, een \textbf{casus}. Het is belangrijk om je onderwerp goed af te bakenen: je gaat voor die \textit{ene specifieke probleemsituatie} op zoek naar een goede oplossing, op basis van de huidige kennis in het vakgebied.

%De doelgroep moet ook concreet en duidelijk zijn, dus geen algemene of vaag gedefinieerde groepen zoals \emph{bedrijven}, \emph{developers}, \emph{Vlamingen}, enz. Je richt je in elk geval op it-professionals, een bachelorproef is geen populariserende tekst. Eén specifiek bedrijf (die te maken hebben met een concrete probleemsituatie) is dus beter dan \emph{bedrijven} in het algemeen.

%Formuleer duidelijk de onderzoeksvraag! De begeleiders lezen nog steeds te veel voorstellen waarin we geen onderzoeksvraag terugvinden.

%Schrijf ook iets over de doelstelling. Wat zie je als het concrete eindresultaat van je onderzoek, naast de uitgeschreven scriptie? Is het een proof-of-concept, een rapport met aanbevelingen, \ldots Met welk eindresultaat kan je je bachelorproef als een succes beschouwen?

%---------- Stand van zaken ---------------------------------------------------

\section{Literatuurstudie}%
\label{sec:literatuurstudie}

Op de Werelddag voor Geestelijke Gezondheid van het jaar 2025, leed er 29\% van Europese medewerkers aan stress of depressie \autocite{EU‑OSHA2025}. Als er enkel naar de IT'ers wordt gekeken, verdubbelt dat cijfer. In het algemeen wordt stress veroorzaakt door een te hoge werkdruk. Een andere oorzaak zoals beslissingsvrijheid kan ook een rol spelen, aangezien werknemers vaak geen kans krijgen om te kiezen welke opdracht ze willen uitvoeren.

\vspace{1em}Er zijn verschillende redenen waarom werknemers in de IT-sector sneller stress voelen dan andere sectoren. Ten eerste verandert technologie razendsnel, waardoor IT'ers voortdurend moeten bijleren om relevant te blijven in hun vak. Deze constante nood aan bijscholing kan stressverhogend werken. Ten tweede is er een grote verantwoordelijkheid binnen de functie: fouten kunnen leiden tot datalekken, financiële schade of stilgevallen bedrijfsprocessen, wat de druk sterk verhoogt. Ten derde is het werk nooit 'af': er zijn wel altijd nieuwe bugs of beveiligingsrisico's. Ten vierde is er een gebrek aan begeleiding voor een kwart van de IT professionals bij het betreden van de sector. Enkel 15\% van de Europese IT professionals heeft een mentor \autocite{ISACA2025}.Ten laatste is er een hoge werkdruk, doordat projecten strikte deadlines hebben en systemen continu moeten blijven functioneren. Samen tonen al deze redenen aan dat werken in de IT-sector zeker een stress-gevende job is.

\vspace{1em}Werkstress veroorzaakt naast ernstige gevolgen voor de mentale gezondheid en werkprestaties ook vaak cognitieve problemen: concentratie- en geheugenstoornissen. Wanneer al deze gevolgen aanhouden, vergroot de kans op een burnout. Dit zijn ook nadelen voor de werkgever: de productiviteit daalt, de foutgevoeligheid stijgt, en de kans dat de werknemer ziek meldt, neemt toe. Bovendien wordt hierdoor niet alleen de werktevredenheid en motivatie negatief beïnvloed, maar ook de teamdynamiek en organisatiesfeer.
%Hier beschrijf je de \emph{state-of-the-art} rondom je gekozen onderzoeksdomein, d.w.z.\ een inleidende, doorlopende tekst over het onderzoeksdomein van je bachelorproef. Je steunt daarbij heel sterk op de professionele \emph{vakliteratuur}, en niet zozeer op populariserende teksten voor een breed publiek. Wat is de huidige stand van zaken in dit domein, en wat zijn nog eventuele open vragen (die misschien de aanleiding waren tot je onderzoeksvraag!)?

%Je mag de titel van deze sectie ook aanpassen (literatuurstudie, stand van zaken, enz.). Zijn er al gelijkaardige onderzoeken gevoerd? Wat concluderen ze? Wat is het verschil met jouw onderzoek?

%Verwijs bij elke introductie van een term of bewering over het domein naar de vakliteratuur, bijvoorbeeld~\autocite{Hykes2013}! Denk zeker goed na welke werken je refereert en waarom.

%Draag zorg voor correcte literatuurverwijzingen! Een bronvermelding hoort thuis \emph{binnen} de zin waar je je op die bron baseert, dus niet er buiten! Maak meteen een verwijzing als je gebruik maakt van een bron. Doe dit dus \emph{niet} aan het einde van een lange paragraaf. Baseer nooit teveel aansluitende tekst op eenzelfde bron.

%Als je informatie over bronnen verzamelt in JabRef, zorg er dan voor dat alle nodige info aanwezig is om de bron terug te vinden (zoals uitvoerig besproken in de lessen Research Methods).

% Voor literatuurverwijzingen zijn er twee belangrijke commando's:
% \autocite{KEY} => (Auteur, jaartal) Gebruik dit als de naam van de auteur
%   geen onderdeel is van de zin.
% \textcite{KEY} => Auteur (jaartal)  Gebruik dit als de auteursnaam wel een
%   functie heeft in de zin (bv. ``Uit onderzoek door Doll & Hill (1954) bleek
%   ...'')

%Je mag deze sectie nog verder onderverdelen in subsecties als dit de structuur van de tekst kan verduidelijken.

%---------- Methodologie ------------------------------------------------------
\section{Methodologie}
\label{sec:methodologie}

Het ontwerpen van een AI-systeem vereist een gestructureerde aanpak. Deze proef volgt een methodologie die bestaat uit vijf delen:
\begin{itemize}
    \item requirements-analyse - 1 week
    \item literatuurstudie - 3 weken
    \item haalbaarheidsonderzoek -3 weken
    \item analyse van benodigde ontwikkelresources + ontwikkelingy - 11 weken
    \item risico- en impactanalyse - 1 week
\end{itemize}

\vspace{1em}De eerste stap in het onderzoek is een requirements-analyse. Hierin wordt vastgesteld welke functies het AI-systeem allemaal moet kunnen vervullen. Er wordt ook geanalyseerd welke soorten data hiervoor nodig zijn. Door deze analyse ontstaat een gedetailleerd overzicht van de functionele en niet-functionele eisen van het systeem.

\vspace{1em}Vervolgens gebeurt er een literatuurstudie om beter inzicht te krijgen in bestaande systemen die vergelijkbare doelen nastreven. Eerst wordt er onderzoek gedaan naar hoe de taakallocatie 'nu' gebeurt en of er al een Ai-systeem of iets soortgelijks voor bestaat.

Vervolgens wordt er onderzocht hoe een AI-systeem tot stand komt. Aan de hand van deze resultaten wordt er een conceptueel model gemaakt. Dit wordt tot stand gebracht door een UML tool, zoals Enterprise Architect. Eenmaal dat model is aangemaakt, kan er een prototype gemaakt worden.

\vspace{1em}Nadien wordt de haalbaarheid van het systeem onderzocht. Hierbij worden beschikbare middelen zoals tijd, budget en technische expertise in kaart gebracht. Deze fase helpt ook bij het prioriteren van systeemfuncties en het inschatten van potentiële obstakels bij implementatie.

\vspace{1em}Er wordt een gedetailleerde analyse uitgevoerd van de benodigde ontwikkelresources. Hierbij wordt gekeken naar de technische componenten (zoals softwaretools en frameworks). Op basis hiervan wordt de meest geschikte technologie gekozen. Als eerste stap wordt een degelijke Proof of Concept (PoC) uitgewerkt, gebaseerd op een regel- en score-gebaseerd systeem. Dit systeem wordt geïmplementeerd via een gestructureerde technologie-stack, met Python als kerntechnologie en PostgreSQL voor de datalaag.

\vspace{1em}Tot slot wordt een risico- en impactanalyse uitgevoerd om mogelijke technische, organisatorische en ethische uitdagingen in kaart te brengen. Hierbij wordt er gekeken naar risico’s zoals privacy en dataveiligheid.

%Hier beschrijf je hoe je van plan bent het onderzoek te voeren. Welke onderzoekstechniek ga je toepassen om elk van je onderzoeksvragen te beantwoorden? Gebruik je hiervoor literatuurstudie, interviews met belanghebbenden (bv.~voor requirements-analyse), experimenten, simulaties, vergelijkende studie, risico-analyse, PoC, \ldots?

%Valt je onderwerp onder één van de typische soorten bachelorproeven die besproken zijn in de lessen Research Methods (bv.\ vergelijkende studie of risico-analyse)? Zorg er dan ook voor dat we duidelijk de verschillende stappen terug vinden die we verwachten in dit soort onderzoek!

%Vermijd onderzoekstechnieken die geen objectieve, meetbare resultaten kunnen opleveren. Enquêtes, bijvoorbeeld, zijn voor een bachelorproef informatica meestal \textbf{niet geschikt}. De antwoorden zijn eerder meningen dan feiten en in de praktijk blijkt het ook bijzonder moeilijk om voldoende respondenten te vinden. Studenten die een enquête willen voeren, hebben meestal ook geen goede definitie van de populatie, waardoor ook niet kan aangetoond worden dat eventuele resultaten representatief zijn.

%Uit dit onderdeel moet duidelijk naar voor komen dat je bachelorproef ook technisch voldoen\-de diepgang zal bevatten. Het zou niet kloppen als een bachelorproef informatica ook door bv.\ een student marketing zou kunnen uitgevoerd worden.

%Je beschrijft ook al welke tools (hardware, software, diensten, \ldots) je denkt hiervoor te gebruiken of te ontwikkelen.

%Probeer ook een tijdschatting te maken. Hoe lang zal je met elke fase van je onderzoek bezig zijn en wat zijn de concrete \emph{deliverables} in elke fase?

%---------- Verwachte resultaten ----------------------------------------------
\section{Verwacht resultaat, conclusie}%
\label{sec:verwachte_resultaten}

In dit onderzoek wordt een duidelijk beeld verwacht van hoe een AI-gestuurd takensysteem kan bijdragen aan het verminderen van werkgerelateerde stress binnen de IT-sector (Cisco). Het uiteindelijke resultaat is een conceptueel onderbouwd model (+ prototype) dat toont hoe taken dynamisch kunnen worden verdeeld op basis van zowel objectieve IT-logdata als korte en vrijwillige taakreflecties van medewerkers.

Voor de doelgroep (IT-werknemers van Cisco )betekent dit een evenwichtigere werkdag. Dit vermindert de kans op langdurige stress en burnouts. Voor Cisco biedt het model inzicht in hoe een intelligent taakverdelingssysteem kan bijdragen aan lagere foutpercentages en hogere productiviteit.

%Hier beschrijf je welke resultaten je verwacht. Als je metingen en simulaties uitvoert, kan je hier al mock-ups maken van de grafieken samen met de verwachte conclusies. Benoem zeker al je assen en de onderdelen van de grafiek die je gaat gebruiken. Dit zorgt ervoor dat je concreet weet welk soort data je moet verzamelen en hoe je die moet meten.

%Wat heeft de doelgroep van je onderzoek aan het resultaat? Op welke manier zorgt jouw bachelorproef voor een meerwaarde?

%Hier beschrijf je wat je verwacht uit je onderzoek, met de motivatie waarom. Het is \textbf{niet} erg indien uit je onderzoek andere resultaten en conclusies vloeien dan dat je hier beschrijft: het is dan juist interessant om te onderzoeken waarom jouw hypothesen niet overeenkomen met de resultaten.

